\documentclass{cmc}
\usepackage{multirow}

\begin{document}

\pagestyle{fancy}
\lhead{\textit{\textbf{Computational Motor Control, Spring 2018} \\
    Python exercise, Lab 9, BONUS ASSIGNMENT}} \rhead{Student \\
  Names}

\section*{Student names: \ldots (please update)}

\textit{Instructions: The goal of this session is to get you
  familiarized with the NeuroRobotics Platform (NRP) developed by the
  Human Brain Project.  During the session you will work with the NRP
  on two experiments. Namely, a pendulum model and a mouse model. You
  will have to use the features in the platform to complete simple
  tasks described below and provide solutions. Also you will be
  answering questions regarding the usability and features of the
  platform as you perform the experiments. \textbf{\corr{You only need
      to submit a small report answering the questions described in
      this below. The report should be no longer than 3 pages in
      total}} Submit a single document
  \corr{NRP\_report\_name1\_name2\_name3.zip} where name\# are the
  team member’s last names.  \corr{Please submit only one report per
    team!}}  \\ {\corr{ This assignment gives a 0.25 (if carried out
    reasonably well) or 0.5 (if carried out very well) bonus point on
    the grade for the mini project. For instance, your mini project
    grade can be increased from 5.0 to 5.5. Deadline May 18, 23h55. }
  \\

  \section*{Pendulum Experiment}
  \label{sec:pendulum-experiment}

  This experiment contains the model of a simple pendulum actuated by
  a pair of antagonist muscles. It is similar to the model you have
  developed and worked with during \textit{Lab 4} and \textit{Lab
    5}. The pendulum model is developed to replicate the behavior of
  mouse hind limb femur and pelvis segments with a single degree of
  freedom at the hip joint. Hence, the physical properties for the
  pendulum model are extracted from the mouse data. Table
  \ref{tab:pendulum_properties} describes the physical properties of
  the pendulum used in the simulation. The model contains two muscles,
  one on each side of the joint. The muscle and their functions on the
  joint is described in \ref{tab:muscle_properties}. For properties of
  the muscles you can have a look at the
  \corr{pendulum\_properties.json} in the Experiment Files section on
  the NRP.


  \textit{Note : The physics of the model is scaled to avoid numerical
  instabilities. Please refer to Lab 7 for more details on physics scaling}

  \begin{table}[H]
    \centering
    \begin{tabular}{|c|c|c|c|c|c|}
      \hline		
      \multirow{2}{*}{\textbf{Segment}} & \multirow{2}{*}{\textbf{{Mass}[$g$]}}
      &    \multirow{2}{*}{\textbf{{Length }[$cm$]}} & \multicolumn{3}{|c|}{\textbf{{Moment of Inertia}}} \\
                                        &          &        & \textbf{Ixx} [$g$ $cm^2$] & \textbf{Iyy} [$g$ $cm^2$] & \textbf{Izz}[$g$ $cm^2$] \\ \hline
      Pendulum  & 1.11126 & 1 & 9.1                  & 4.24                  & 9.10                 \\ \hline
    \end{tabular}
    \caption{Physical Properties of the Pendulum}
    \label{tab:pendulum_properties}
  \end{table}

	\begin{table}[H]
          \centering
          \begin{tabular}{|c|c|c|c|c|c|}
            \hline
            \textbf{Muscle}       & \textbf{Abbreviation} & \textbf{Type } & \textbf{Hip} & \textbf{Knee} & \textbf{Ankle} \\ \hline
            Psoas Major           & PMA                   & Mono-articular & Flexion      & -             & -              \\ \hline
            Caudofemoralis        & CF                    & Mono-articular & Extension    & -             & -              \\ \hline
          \end{tabular}
          \caption{Flexor and Extensor muscles in the pendulum model}
          \label{tab:muscle_properties}
	\end{table}
        
        \newpage
        \subsection*{Questions}
        \label{sec:questions}

        Launch the pendulum model and open Editor function and then
        navigate to the Transfer Functions tab. Here you will find the
        three main python files that allow you to control and modify
        the experiment.
        \begin{itemize}
        \item \textbf{\corr{apply\_torque.py}} : Transfer function to
          apply the computed joint torques to the Gazebo simulation.
        \item \textbf{\corr{compute\_muscle\_torque.py}} : Transfer
          function to step the musculoskeletal model and compute the
          joint torques.  This is the function you need to work with
          to change the muscle activation patterns while answering the
          questions.
        \item \textbf{\corr{save\_csv\_joint\_state.py}} : Transfer
          function to save the data to a csv file.
        \end{itemize}
        
        \subsubsection*{1a. Briefly describe the behavior of the
          pendulum with the default parameters, e.g. in terms of
          frequency and amplitude of oscillations.  Plots are not
          necessary.}
        \label{sec:1a}

        \subsubsection*{1b. Is the pendulum in a limit cycle behavior?
          What feature from the NRP did you use to check the limit
          cycle behavior?}

        \subsubsection*{1c. In \corr{compute\_muscle\_torque.py}
          change the muscle activation's frequency of the flexor and
          extensor by the same factor. Does the pendulum frequency
          correlate with the applied frequency of the muscle
          activation's?}
        \label{sec:1c}

        \subsubsection*{1d. Download the csv data file from the
          Experiment files folder at the end of a simulation. Read the
          data externally using Python(For example with Spyder) or any
          other program of your choice. Show a plot containing the
          joint states (position and velocity) as a function of time
          and phase plot containing joint position versus velocity}
        \label{sec:1d}

        \newpage

        \section*{Mouse Experiment}
      
        \label{sec:mouse-experiment}

        For the mini project you have been working on the mouse
        locomotion based purely on reflexes and feed backs.  The
        current mouse model(which is part of current research)
        contains a bio-inspired Central Pattern Generator network
        developed by [1]. The physical properties and the
        musculoskeletal parameters of the model are the same as you
        have seen for the mini project.

        \subsection*{Questions}

        \subsubsection*{2a. Clone the mouse model to run the
          experiment. Translate the mouse model by -2 m in the world
          before starting the simulation.Then start the simulation and
          observe the behavior. Describe the gait generated by the
          hind limbs with default parameters. Eg. in terms of duty
          factor, and phase relationship between left and right legs}

        \subsubsection*{2b. Is the mouse model behaving in a
          qualitatively similar manner to the Webots model?. Briefly
          discuss similarities and differences}

        \subsubsection*{2c. What happens if the mouse model is started
          in the air rather than on ground? Try different starting
          heights, and briefly discuss}

        \subsubsection*{2d. During the mouse locomotion, interactively
          add objects into the simulation to create a perturbation and
          explain how the model behaves in this case. No, need of any
          plots. Provide a screen shot of the mouse with the inserted
          objects.}

        \subsubsection*{2e. Using the logged csv data plot the muscle
          activation's of all the muscles in either left or right hind
          limb by reading the csv file externally using Python or
          other program of your choice. Briefly describe the muscle
          activation pattern you observe in the plot. }
      
        \section*{Neuro-Robotics Platform}
        \label{sec:nrp-feedb-quest}

        \subsection*{Questions}
        
        \subsubsection*{3a. Do you like the idea of accessing the
          neuromechanical models on remote servers through a web
          interface?}

        \subsubsection*{3b. How do you rate the graphical user
          interface? Any suggestions on how to improve it?}

        \subsubsection*{3c.How do you rate the programming
          environment?  (Obviously there was little time to really
          test it today). Any suggestions on how to improve it?}

        \subsubsection*{3d. Please also complete the following online
          feedback
          \href{https://docs.google.com/forms/d/e/1FAIpQLSfmcBCIoLaNXOuNBGXvWdIbC453-T6F-mEafyHQbhINUy8IMw/viewform}{\corr{form}
          }}

        \section*{REFERENCES}
        \label{sec:references}

        1. \textit{Central control of inter limb coordination and
          speed-dependent gait expression in quadrupeds.} : Danner,
        S. M., Wilshin, S. D., Shevtsova, N. A. and Rybak,
        I. A. (2016), J Physiol, 594: 6947-6967. doi:10.1113/JP272787
        \href{https://www.ncbi.nlm.nih.gov/pmc/articles/PMC5134391/pdf/TJP-594-6947.pdf}{\corr{[link]}}}
       
    \end{document}


%%% Local Variables:
%%% mode: latex
%%% TeX-master: t
%%% End:
