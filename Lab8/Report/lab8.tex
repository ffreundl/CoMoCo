
\documentclass{cmc}

\begin{document}

\pagestyle{fancy}
\lhead{\textit{\textbf{Computational Motor Control, Spring 2018} \\
    Python exercise, Lab 8, GRADED}} \rhead{Student \\ Names}

\section*{Student names: \ldots (please update)}

\textit{Instructions: Update this file (or recreate a similar one,
  e.g.\ in Word) to prepare your answers to the questions. Feel free
  to add text, equations and figures as needed. Hand-written notes,
  e.g.\ for the development of equations, can also be included e.g.\
  as pictures (from your cell phone or from a scanner).
  \textbf{\corr{This lab is graded.}} and needs to be submitted before
  the \textbf{\corr{Deadline : 05-06-2018
      Midnight}}.\\
  \textbf{\corr{You only need to submit one final report for all of
      the following exercises combined henceforth.}}\\ Please submit
  both the source file (*.doc/*.tex) and a pdf of your document, as
  well as all the used and updated Python functions in a single zipped
  file called \corr{final\_report\_name1\_name2\_name3.zip} where
  name\# are the team member’s last names.  \corr{Please submit only
    one report per team!}}
\\

\subsection*{Locomotion}
\label{sec:locomotion}

In the previous lab you explored the reflex model with the mouse fixed
in air. By changing the muscle activation gains you were able to
produce a stable gait behavior based on the four phases described by
[1]. In the current model the mouse will no longer be fixed but it can
move around in three dimensions.

Important to note the differences in [1] and the mouse model. The
morphology of mouse and cat are different, the parameters used for
musculo-skeletal properties are completely different from [1]. Do not
be surprised to see results different from [1]. Also the mouse model
is new and currently being developed.  Please report if you find any
bugs and please point out any interesting behaviors that you may
observe during the exercises.

Now you can explore how the reflex model behaves under different
conditions and study the locomotion behavior.

In the current model we explore three important quantities during the
stance phase in the reflex model,
\begin{itemize}
\item Coupling between the legs : This ensures that one of the hind
  legs lifts-off from the ground only when the other leg is in stance
  phase. This coupling can be enabled or disabled by the setting the
  variable \corr{reflexes::COUPLING} to either True or False in the
  \corr{mouse.py} controller file at any instant of time. In the
  current model we tuned the parameters to produce locomotion for the
  coupling case.  During the exercises you will observe how this
  influences the overall model.
\item Hip extension rule : During the stance phase, this reflex
  parameter determines the angle of the hip extension at which the
  stance phase terminates and enters the lift-off phase. This reflex
  parameter can be enabled or disabled by the variable
  \corr{reflexes::HIP\_EXTENSION\_RULE} to either True or False in the
  \corr{mouse.py} controller file at any instant of time.
\item Ankle unloading rule : During the stance phase, this reflex
  parameter terminates the stance phase when the force in soleus
  muscle drops below a threshold value and enters the lift-off
  phase. This reflex parameter can be enabled or disabled by the
  variable \corr{reflexes::ANKLE\_\-UNLOADING\_\-RULE} to either True or
  False in the \corr{mouse.py} controller file at any instant of time.
\end{itemize}

Download the updated model from the git repository along with the
controller files. In the new files, you can replace the muscle
activation constants that you tuned from the previous lab in the
current \corr{reflexes.py} file.

\newpage
\textbf{Data logging} The current model now saves additional entities
during the simulation. They are,
\begin{itemize}
\item Trajectory of the left and right right hind foot
\item Ground contact detection for the left and right hind foot
\item Muscle activations and tendon forces from left and right hind
  limb muscles
\item Joint angular positions from left and right hind limb
\end{itemize}

Use \corr{Results::load\_data.py} file to read and plot the data saved
in \corr{Results}

\section*{\textnormal{Questions}}
\label{sec:questions}

\subsection*{7c. With the previously tuned reflex model, report if you
  observe a stable gait or not with the 3D model for at least 2
  seconds of simulation time? If you do not observe a stable gait then
  tune the muscle activation constants (Kx) in the \corr{reflexes.py}
  file to obtain stable gait. Report the changes in muscle activation
  constants if any. Plot and explain the joint angles of hip, knee and
  ankle joints from left and right hind limbs. Plot the muscle
  activations of either left or right hind limb. Explain how the
  muscle activity correlates with the different phases of gait. Plot
  the ground contact of both left and right hing feet. Use
  \corr{Results::load\_data.py} files to see how to load the data at
  the end of the simulation.  Use subplots to make your plots and plot
  both left and right joint angles in each subplot. See the example
  given in \corr{Results::load\_data.py} (Note : You need to press
  revert button in Webots to save the data.). Also compute and report
  the duty factor of the model }

\subsection*{7d. Now explore the coupling between the hind legs of the
  mouse. In \corr{mouse.py} disable coupling at the beginning of the
  simulation.  Plot the joint angles similar to the one before and
  explain the gait you observe without coupling. What happens to the
  gait when you start the simulation with coupling and then turn it
  off/disable it after 1.0 second of simulation?.  To do this you can
  use an \textit{if} condition in the \corr{mouse.py::run} and you can
  obtain the current simulation time using the Webots robot class
  method \corr{self.getTime()}.  Why is there a difference in behavior
  in the above two cases and can it be improved by re-tuning the
  parameters in the model?}

\subsection*{7e. Explore the robustness of the model with coupling
  under different terrains and small perturbations. Use
  \corr{cmc\_mouse\_uphill.wbt} and \corr{cmc\_mouse\_downhill.wbt}
  world files to simulate uphill and downhill climbing scenarios. In
  the world files you can change
  the inclination of slope by changing the rotation field in \\
  Floor Solid $\rightarrow$ Floor Group $\rightarrow$ Hill
  Transformation $\rightarrow$ rotation. Report the different
  conditions under which the model is explored. Show one set of joint
  angles plot and ground contact plot for one condition of up hill and
  down hill scenario. With the same plots explain how the gait
  frequency changes with respect to the terrain. Apply small
  perturbations to the model using the Webots GUI and show how the
  gait recovers using the ground contact plots of hind limbs.}

\subsection*{As mentioned earlier, two main reflexes used in the model
  for stance termination are hip extension rule and ankle unloading
  rule. To explore, the role of each we start the simulation with all
  couplings and reflexes enabled. At time == 1. we can start disabling
  each of the rules and then look at the behavior of the model}

\subsection*{7f. Is one type of reflex rule more important than the
  other to generate stable locomotion? To explore the differences
  first disable the hip extension rule at 1 second of simulation time
  and observe the models gait. Do the same by disabling the ankle
  loading rule at 1 second of simulation time and observe the models
  gait. If you do not observe any differences in gait can you explain
  how the coupling parameter is responsible for the stability?
  Disable the coupling parameter to see the models performance. Would
  re-tuning the model parameters help in stabilizing the model without
  coupling? }
\newpage
\subsection*{7g. How do the reflex rules perform under different
  terrains.  Use \\ \corr{cmc\_mouse\_uphill.wbt} and
  \corr{cmc\_mouse\_downhill.wbt} world files to simulate uphill and
  downhill climbing scenarios. Use the default slope of $0.08$ rad.
  Show the joint trajectories and ground contacts while each of the
  reflex parameters is disabled. Using the observations from 8b
  describe if having the combined reflex rule helps in stabilizing the
  model better.}

\subsection*{In the coming weeks the goal will be to explore the above
  scientific questions but also now by varying other the reflex
  parameters in the system and tuning to the model to perform better
  even under uncoupled condition. More details will be discussed
  later}



\section*{Discussion Questions}
\label{sec:discussion-quesions}

These questions require only a discussion and no implementation of any
model is expected.

\subsection*{a. What would be the potential benefits of adding a CPG
  to the locomotor circuit? Try to relate to literature as much as
  possible.}
\label{sec:a}

\subsection*{b. How could the model be extended? (e.g. more or less
  detailed models, 3D, full body, more sensory modalities, inclusion
  of more brain parts, ..., discuss, no need to implement). Try to
  relate to literature as much as possible.}
\label{sec:b}

\subsection*{c. Which type of additional experiments could be done
  (with the current model, and with extended models) and what would be
  your predictions? Discuss, no need to implement. Try to relate to
  literature as much as possible.}
\label{sec:c}



\section*{REFERENCES}
\label{sec:references}

1. \textit{Computer Simulation of Stepping in the Hind Legs of the
  Cat: An Examination of Mechanisms Regulating the Stance-to-Swing
  Transition} : Orjan Ekeberg and Keir Pearson. Journal of
Neurophysiology 2005 94:6, 4256-4268
\href{https://www.physiology.org/doi/abs/10.1152/jn.00065.2005}{\corr{download
    link}}


\end{document}

%%% Local Variables:
%%% mode: latex
%%% TeX-master: t
%%% End:
